令 $\mathcal{X}=\{S: S\subseteq [0,1]\mbox{ and } |S|=n \}$. 

若以下条件满足,则$f:\mathcal{X}\rightarrow \mathbb{R}$ 是一个$\mathcal{X}$上关于赫斯多夫距离$d_H(\cdot, \cdot)$的连续函数

$\forall \epsilon > 0, \exists \delta >0$, 对于任意一个 $S, S'\in\mathcal{X}$, if $d_H(S, S') < \delta$, then $|f(S)-f(S')|< \epsilon$.

我们证明了 $f$ 可以被对称函数和连续函数的复合函数任意逼近

% \begin{theorem}
\begin{customthm}{1}
假设 $f:\mathcal{X}\rightarrow \mathbb{R}$ 是一个关于赫斯多夫距离$d_H(\cdot, \cdot)$的连续集函数。 $\forall \epsilon > 0$, $\exists$ 一个连续函数 $h$ 和一个对称函数 $g(x_1, \dots, x_n)=\gamma \circ \mbox{MAX}$, 其中 $\gamma$ 一个连续函数, $\mbox{MAX}$是一个取 $n$ 个向量最为输入,然后返回一个元素层面上的最大值的最大化向量运算符, 对于任意一个 $S\in\mathcal{X}$,
% Formally, $\forall \epsilon > 0$, $\exists g_{\epsilon}:\underbrace{\R^{K_\epsilon}\times \dots \times \R^{K_\epsilon}}_n\rightarrow \R$ which is a symmetric function and $h_{\epsilon}: \R^3 \rightarrow \R^{K_\epsilon}$ such that 
\begin{align*}
	|f(S) - \gamma(\mbox{MAX}(h(x_1), \ldots, h(x_n)))| < \epsilon
\end{align*}
其中 $x_1, \ldots, x_n$ 是按指定顺序从 $S$ 中提取的元素。
\end{customthm}
%\end{theorem}

\begin{proof}
通过 $f$ 的连续性, 引入 $\delta_{\epsilon}$ 来说明 
$|f(S)-f(S')|<\epsilon$ 对于任意一个 $S, S'\in \mathcal{X} \mbox{ if } d_H(S, S')<\delta_{\epsilon}$. 

定义 $K=\lceil 1/\delta_{\epsilon}\rceil$, 将 $[0,1]$ 均分成 $K$ 个间隔,同时定义辅助函数,将点映射到它所在间隔的左端:
$$\sigma(x)=\frac{\lfloor K x \rfloor}{K}$$
令 $\tilde{S}=\{\sigma(x):x\in S\}$
$$|f(S)-f(\tilde{S})|< \epsilon$$
因为 $d_H(S, \tilde{S})<1/K\le \delta_{\epsilon}$.

令 $h_k(x)=e^{-d(x, [\frac{k-1}{K}, \frac{k}{K}])}$ 为一个轻指示函数,其中 $d(x, I)$ 是点的间隔距离。 令 $\myvec h(x)=[h_1(x); \ldots; h_K(x)]$,  $\myvec h:\mathbb{R}\rightarrow \mathbb{R}^K$. 

令 $v_j(x_1, \ldots, x_n)=\max\{\tilde{h}_j(x_1),\ldots,\tilde{h}_j(x_n)\}$ 代表第$j$-th区间中$S$中点的占比。 令 $\myvec v=[v_1;\ldots; v_K]$, $\myvec v:\underbrace{\mathbb{R}\times \ldots\times \mathbb{R}}_{n}\rightarrow \{0, 1\}^K$ 是一个对称函数, 代表每个区间中$S$中点的占比。

定义 $\tau:\{0, 1\}^K\rightarrow \mathcal{X}$ 为 $\tau(v)=\{\frac{k-1}{K}: v_k\ge 1\}$, 代表占用向量到表示每个占用区间左端集合的映射。 易得:
\begin{align*}
\tau(\myvec v(x_1, \ldots, x_n))\equiv \tilde{S} 
\end{align*}
其中 $x_1, \ldots, x_n$ 是按指定顺序从 $S$ 中提取的元素。

令 $\gamma:\mathbb{R}^K\rightarrow \mathbb{R}$ 为一个连续函数,对于 $v\in\{0, 1\}^K$ 有 $\gamma(\myvec v)=f(\tau(\myvec v))$ 对于 $v\in\{0, 1\}^K$. 则
\begin{align*}
&|\gamma(\myvec v(x_1, \ldots, x_n))-f(S)|\\
=&|f(\tau(\myvec v(x_1, \ldots, x_n)))-f(S)|<\epsilon
\end{align*}

注意 $\gamma(\myvec v(x_1, \ldots, x_n))$ 可以被写成以下形式:
\begin{align*}
\gamma(\myvec v(x_1, \ldots, x_n))=&\gamma(\mbox{MAX}(\myvec h(x_1), \ldots, \myvec h(x_n)))\\
=&(\gamma \circ \mbox{MAX}) (\myvec h(x_1),\ldots,\myvec h(x_n))
\end{align*}
显然 $\gamma \circ \mbox{MAX}$ 是一个对称函数。
% $f(\tau(\myvec v(x_1,\ldots, x_n)))$. Clearly, $\gamma$ is a piece-wise constant function. Due to the universal approximation property of neural networks, 
\end{proof}

接下来给出定理2的证明。
定义 $\myvec u=\underset{x_i\in S}{\mbox{MAX}}\{h(x_i)\}$ 为$f$的子网络,表示$[0,1]^m$上点集到$K$-维向量的映射。以下定理表明输入点集中较少点缺失或者额外微小点扰动不会改变我们网络的输出:
% \begin{theorem}
\begin{customthm}{2}
假设 $\myvec u:\mathcal{X}\rightarrow \mathbb{R}^K$ 其中 $\myvec u=\underset{x_i\in S}{\mbox{MAX}}\{h(x_i)\}$ 且 $f=\gamma \circ \myvec u$. 则, 
\begin{enumerate}[label=(\alph*)]   
    \item $\forall S, \exists~\mathcal{C}_S, \mathcal{N}_S\subseteq \mathcal{X}$,  $f(T)=f(S)$ if  $\mathcal{C}_S\subseteq T\subseteq \mathcal{N}_S$;
    %\item Define the equivalence relation $\sim$ as $S\sim S'$ if $\myvec u(S)=\myvec u(S')$, then $\mathcal{C}_S=\underset{S'\sim S}{\cap} S'$ and $\mathcal{N}_S=\underset{S'\sim S}{\cup} S'$.
    % \item Let $S\sim S'$ if $\myvec u(S)=\myvec u(S')$. $\mathcal{C}_S=\underset{S'\sim S}{\cap} S'$, $\mathcal{N}_S=\underset{S'\sim S}{\cup} S'$.
    \item $|\mathcal{C}_S| \le K$
\end{enumerate}
%\label{thm:thm2}
%     \begin{itemize}            
%     \item Critical points.
%     \\$\mathbb{C}=\{x_i: \myvec u_j(S)=h_j(x_i) \mbox{ for some } 1\le j\le K\}\cap S$
%     \item Free-space points.
%     \\$\mathbb{F}=\{x_i: \myvec u_j(S)<h_j(x_i) \mbox{ for some } 1\le j\le K\}\cap S$
%     \item Non-critical points. 
%     \\$\mathbb{N}=\{x_i: \myvec u_j(S)>h_j(x_i) \mbox{ for some } 1\le j\le K\}\cap S$
% For any $S'$ such that  $\mathbb{C} \subseteq S' \subseteq \mathbb{C}\cup\mathbb{N}$, $\myvec u(S')\equiv \myvec u(S)$; for any $S'$ such that $S'\cap \mathbb{F}\neq \emptyset$, $\myvec u(S')\neq \myvec u(S)$.
%\end{itemize}
\end{customthm}
%\end{theorem}
\begin{proof}
显然, $\forall S\in \mathcal{X}$, $f(S)$ 由$\myvec u(S)$所决定。所以我们只需要证明 
$\forall S, \exists\,\mathcal{C}_S, \mathcal{N}_S\subseteq \mathcal{X}, f(T)=f(S)\,\mbox{if}\,\mathcal{C}_S\subseteq T\subseteq \mathcal{N}_S$. 

对于输出向量$\myvec u$的第$j$维 , 存在至少一个 $x_j \in \mathcal{X}$ 使得 $h_j(x_j)=\myvec u_j$, 其中 $h_j$ 是$h$中第$j$维输出向量。 将 $\mathcal{C}_S$ 作为所有$x_j$ , $j=1,\ldots,K$ 的聚合,则 $\mathcal{C}_S$ 满足上述条件。

因此对于$f$,附加任意额外的点$x$使得$\mathcal{C}_S$的每一维满足$h(x)\le \myvec u(S)$ 而不改变$\myvec u$。从而得出,可以把所有这样点添加到$\mathcal{N}_S$来得到$\mathcal{T}_S$。




\end{proof}